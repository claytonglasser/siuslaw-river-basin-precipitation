\documentclass[]{article}
\usepackage{lmodern}
\usepackage{amssymb,amsmath}
\usepackage{ifxetex,ifluatex}
\usepackage{fixltx2e} % provides \textsubscript
\ifnum 0\ifxetex 1\fi\ifluatex 1\fi=0 % if pdftex
  \usepackage[T1]{fontenc}
  \usepackage[utf8]{inputenc}
\else % if luatex or xelatex
  \ifxetex
    \usepackage{mathspec}
  \else
    \usepackage{fontspec}
  \fi
  \defaultfontfeatures{Ligatures=TeX,Scale=MatchLowercase}
\fi
% use upquote if available, for straight quotes in verbatim environments
\IfFileExists{upquote.sty}{\usepackage{upquote}}{}
% use microtype if available
\IfFileExists{microtype.sty}{%
\usepackage{microtype}
\UseMicrotypeSet[protrusion]{basicmath} % disable protrusion for tt fonts
}{}
\usepackage[margin=1in]{geometry}
\usepackage{hyperref}
\hypersetup{unicode=true,
            pdftitle={Siuslaw Basin Precipitation Analysis Proposal},
            pdfauthor={Clayton Glasser},
            pdfborder={0 0 0},
            breaklinks=true}
\urlstyle{same}  % don't use monospace font for urls
\usepackage{graphicx,grffile}
\makeatletter
\def\maxwidth{\ifdim\Gin@nat@width>\linewidth\linewidth\else\Gin@nat@width\fi}
\def\maxheight{\ifdim\Gin@nat@height>\textheight\textheight\else\Gin@nat@height\fi}
\makeatother
% Scale images if necessary, so that they will not overflow the page
% margins by default, and it is still possible to overwrite the defaults
% using explicit options in \includegraphics[width, height, ...]{}
\setkeys{Gin}{width=\maxwidth,height=\maxheight,keepaspectratio}
\IfFileExists{parskip.sty}{%
\usepackage{parskip}
}{% else
\setlength{\parindent}{0pt}
\setlength{\parskip}{6pt plus 2pt minus 1pt}
}
\setlength{\emergencystretch}{3em}  % prevent overfull lines
\providecommand{\tightlist}{%
  \setlength{\itemsep}{0pt}\setlength{\parskip}{0pt}}
\setcounter{secnumdepth}{0}
% Redefines (sub)paragraphs to behave more like sections
\ifx\paragraph\undefined\else
\let\oldparagraph\paragraph
\renewcommand{\paragraph}[1]{\oldparagraph{#1}\mbox{}}
\fi
\ifx\subparagraph\undefined\else
\let\oldsubparagraph\subparagraph
\renewcommand{\subparagraph}[1]{\oldsubparagraph{#1}\mbox{}}
\fi

%%% Use protect on footnotes to avoid problems with footnotes in titles
\let\rmarkdownfootnote\footnote%
\def\footnote{\protect\rmarkdownfootnote}

%%% Change title format to be more compact
\usepackage{titling}

% Create subtitle command for use in maketitle
\newcommand{\subtitle}[1]{
  \posttitle{
    \begin{center}\large#1\end{center}
    }
}

\setlength{\droptitle}{-2em}

  \title{Siuslaw Basin Precipitation Analysis Proposal}
    \pretitle{\vspace{\droptitle}\centering\huge}
  \posttitle{\par}
    \author{Clayton Glasser}
    \preauthor{\centering\large\emph}
  \postauthor{\par}
      \predate{\centering\large\emph}
  \postdate{\par}
    \date{June 17, 2018}


\begin{document}
\maketitle

\section{The Challenge}\label{the-challenge}

The 773 square mile Siuslaw River Basin is one of the most important
natural sites in the United States. More timber has been sourced from
the forest contained within this basin over time than any other area of
equivalent size in the country. In the early 20th century, the Siuslaw
river was the also the richest salmon breeding grounds anywhere in the
United States, with local canneries processing as many at 60,000 coho
salmon a year. Over time, as the effects of forestry practices have been
realized, the salmon population has declined.

Common forestry practices often result the channelization of streams and
the erosion of river banks, which lead to less conducive habitat for
salmon. With fewer salmon in the river, less fertility is deposited in
the stream and the surrounding area, which leads to less conducive
habitat for vegetation, including timber. In effect, poor forest
management practices have resulted in both ecological and economic
losses.

Because of the manner in which the stream system and the timber system
are intertwined, a model than accounts for the inputs and outputs of
both is needed in order to optimize the outcome of either one, let alone
both.

Among the challenges facing anyone who wishes to use this kind of model
is access to system-wide information. While it may never be possible to
generate a truly complete model of a system like a forest because of its
continuity with adjacent systems, forests do often admit of one natural
and useful boundary: the drainage basin. Hydrologically, drainage basins
constitute relatively coherent and self-contained systems. It is
possible to collect and model data about drainage basins completely
enough to create a meaningful and accurate insights about its
hydrodynamics.

High-resolution understanding of the movement of water through an
ecosystem like the Siuslaw River Basin is important to both restoration
ecologists and commercial interests. The following are some its
practical applications:

\begin{itemize}
\tightlist
\item
  Information about stream volume levels can be used to determine the
  appropriate frequency, size, and location of check dams.
\item
  Information about the intensity and duration of flood events can be
  used to inform zoning and construction requirements.
\item
  Information about the intensity and duration of precipitation can be
  used to understand the probable locations of landslides and erosion,
  and inform preventive measures such as soil stabilization plantings.
\end{itemize}

\section{The Data}\label{the-data}

In order to acquire insights like those listed above, multiple data sets
will be combined and multiple analytic methods will utilized.

The primary data I plan to use are measurements of precipitation volume.
The constituent datasets will be derived from up to 50 sources
distributed across the basin, each of which will likely be somewhat
different in character. Some sources date back 50 years; some have daily
data; others have minute by minute data. They are likely to contain a
variety of different data, formats, and conventions, but I think the
core of the data will be timestamps and precip volume. Much of this data
is freely available, if obscure, through entities such as

\begin{itemize}
\tightlist
\item
  Bureau of Land Management
\item
  Siuslaw River Basin Watershed Council
\item
  USDA Forest Service
\end{itemize}

Some of the planned data sources are privately held, but can be
retrieved through personal relationships.

Another important data set will be digital elevation models, which can
be used in combination with GIS to create topographic visualizations.
This information is also freely available, often from the same sources.

\section{The Method}\label{the-method}

TBD

\section{The Deliverables}\label{the-deliverables}

TBD


\end{document}
